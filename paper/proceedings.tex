\documentclass{sigchi}

% Use this section to set the ACM copyright statement (e.g. for
% preprints).  Consult the conference website for the camera-ready
% copyright statement.

% Copyright
\CopyrightYear{2016}
%\setcopyright{acmcopyright}
\setcopyright{acmlicensed}
%\setcopyright{rightsretained}
%\setcopyright{usgov}
%\setcopyright{usgovmixed}
%\setcopyright{cagov}
%\setcopyright{cagovmixed}
% DOI
\doi{http://dx.doi.org/10.475/123_4}
% ISBN
\isbn{123-4567-24-567/08/06}
%Conference
\conferenceinfo{CHI'16,}{May 07--12, 2016, San Jose, CA, USA}
%Price
\acmPrice{\$15.00}

% Use this command to override the default ACM copyright statement
% (e.g. for preprints).  Consult the conference website for the
% camera-ready copyright statement.

%% HOW TO OVERRIDE THE DEFAULT COPYRIGHT STRIP --
%% Please note you need to make sure the copy for your specific
%% license is used here!
% \toappear{
% Permission to make digital or hard copies of all or part of this work
% for personal or classroom use is granted without fee provided that
% copies are not made or distributed for profit or commercial advantage
% and that copies bear this notice and the full citation on the first
% page. Copyrights for components of this work owned by others than ACM
% must be honored. Abstracting with credit is permitted. To copy
% otherwise, or republish, to post on servers or to redistribute to
% lists, requires prior specific permission and/or a fee. Request
% permissions from \href{mailto:Permissions@acm.org}{Permissions@acm.org}. \\
% \emph{CHI '16},  May 07--12, 2016, San Jose, CA, USA \\
% ACM xxx-x-xxxx-xxxx-x/xx/xx\ldots \$15.00 \\
% DOI: \url{http://dx.doi.org/xx.xxxx/xxxxxxx.xxxxxxx}
% }

% Arabic page numbers for submission.  Remove this line to eliminate
% page numbers for the camera ready copy
% \pagenumbering{arabic}

%\let\bibsep\relax

%\setcitestyle{numbers,open={[},close={]}}
% Load basic packages
\usepackage{balance}       % to better equalize the last page
\usepackage{graphics}      % for EPS, load graphicx instead 
\usepackage[T1]{fontenc}   % for umlauts and other diaeresis
\usepackage{txfonts}
\usepackage{mathptmx}
\usepackage[pdflang={en-US},pdftex]{hyperref}
\usepackage{color}
\usepackage{booktabs}
\usepackage{textcomp}
%graphic packages
\usepackage{subfigure}

%\usepackage{float}
% Some optional stuff you might like/need.
\usepackage{microtype}        % Improved Tracking and Kerning
% \usepackage[all]{hypcap} 
   % Fixes bug in hyperref caption linking
\usepackage{ccicons}          % Cite your images correctly!
% \usepackage[utf8]{inputenc} % for a UTF8 editor only

% If you want to use todo notes, marginpars etc. during creation of
% your draft document, you have to enable the "chi_draft" option for
% the document class. To do this, change the very first line to:
% "\documentclass[chi_draft]{sigchi}". You can then place todo notes
% by using the "\todo{...}"  command. Make sure to disable the draft
% option again before submitting your final document.
\usepackage{todonotes}

%\usepackage[table]{xcolor}% http://ctan.org/pkg/xcolor
\usepackage{colortbl}

%\usepackage[numbers,sort,square]{natbib}

% Paper metadata (use plain text, for PDF inclusion and later
% re-using, if desired).  Use \emtpyauthor when submitting for review
% so you remain anonymous.
\def\plaintitle{Investigating Hand-Size and Mobile Touch Interactions}
\def\plainauthor{First Author, Second Author, Third Author,
  Fourth Author, Fifth Author, Sixth Author}
\def\emptyauthor{}
\def\plainkeywords{HCI; Hand-Size; Mobile Touch Interactions; Smartphones; Touch Interactions; Evaluation}
\def\plaingeneralterms{Documentation, Standardization}

% llt: Define a global style for URLs, rather that the default one
\makeatletter
\def\url@leostyle{%
  \@ifundefined{selectfont}{
    \def\UrlFont{\sf}
  }{
    \def\UrlFont{\small\bf\ttfamily}
  }}
\makeatother
\urlstyle{leo}

% To make various LaTeX processors do the right thing with page size.
\def\pprw{8.5in}
\def\pprh{11in}
\special{papersize=\pprw,\pprh}
\setlength{\paperwidth}{\pprw}
\setlength{\paperheight}{\pprh}
\setlength{\pdfpagewidth}{\pprw}
\setlength{\pdfpageheight}{\pprh}

% Make sure hyperref comes last of your loaded packages, to give it a
% fighting chance of not being over-written, since its job is to
% redefine many LaTeX commands.
\definecolor{linkColor}{RGB}{6,125,233}
\hypersetup{%
  pdftitle={\plaintitle},
% Use \plainauthor for final version.
%  pdfauthor={\plainauthor},
  pdfauthor={\emptyauthor},
  pdfkeywords={\plainkeywords},
  pdfdisplaydoctitle=true, % For Accessibility
  bookmarksnumbered,
  pdfstartview={FitH},
  colorlinks,
  citecolor=black,
  filecolor=black,
  linkcolor=black,
  urlcolor=linkColor,
  breaklinks=true,
  hypertexnames=false
}
% create a shortcut to typeset table headings
% \newcommand\tabhead[1]{\small\textbf{#1}}

% End of preamble. Here it comes the document.
\begin{document}

\title{\plaintitle}

\numberofauthors{3}
\author{%
  \alignauthor{Iris Figalist\\
    \affaddr{Ludwig-Maximilians-Universit{\"a}t}\\
    \affaddr{Munich, Germany}\\
    \email{I.Figalist@campus.lmu.de}}\\
  \alignauthor{Jonas Mattes\\
    \affaddr{Ludwig-Maximilians-Universit{\"a}t}\\
    \affaddr{Munich, Germany}\\
    \email{J.Mattes@campus.lmu.de}}\\
  \alignauthor{Sarah Prange\\
    \affaddr{Ludwig-Maximilians-Universit{\"a}t}\\
    \affaddr{Munich, Germany}\\
    \email{Sarah.Prange@campus.lmu.de}}\\
}

\maketitle

\begin{abstract}
  The goal of this study was to find out whether it is possible to foresee a user's hand size by analyzing his mobile touch interactions. We collected data about various touch interactions and tried to find correlations to the user's hand size. The results show a relationship between different features of a touch interaction and the hand size. But we also found out that it is harder to find correlations in an natural setting in which the user can adapt his hand position. A task with a predefined hand position, which was also part of our study, offered the highest correlation. We executed a machine learning algorithm with the highest correlating features and achieved an result of 0.75.
\end{abstract}

\category{H.5.m.}{Information Interfaces and Presentation
  (e.g. HCI)}{Miscellaneous} \category{See \url{http://acm.org/about/class/1998/} for the full list of ACM classifiers. This section is required.}{}{}

\keywords{\plainkeywords}

\section{Introduction}
% part and parcel of everyday life
Interaction with your personal mobile device is an individual daily routine. Devices are increasingly smart and have plenty of functions. As today's mobile devices vary highly in their dimensions, interaction has different levels of difficulties. One or two hands might be necessary for different tasks.\\
People have different hand sizes and tend to have different device sizes, although that might not be closely interlinked. Mobile touch interactions differ widely with hand and device size. Besides mobility, personalization is an important aspect. Our smartphone can only be smart based on our personal information, like for example our residence, contacts and browsing habits.\\ %TO DO quelle?!
Research shows another important personalization factor: hand size. What has been investigated a lot so far is the touch behavior of different users (e.g. \cite{parhi2006target}, \cite{buschek2015touchml}). An example is touch precision in relation to the user's hand size \cite{parhi2006target}. These and others show that hand size actually has an influence on mobile touch interactions. Our hand size could be another personalization aspect and make our devices even smarter. If my device knows about my hand size, it could for instance adapt the layout to help us reaching the important interaction elements.\\
This is our motivation to further investigate hand size. In our study, we were aiming at finding the backward conclusion from the user's touch input to his hand size. In order to keep the data as natural as natural as possible, we focused on common touch interactions and left the hand position over to the user. We chose one artificial task with predetermined hand position for control purposes. Finding out the user's hand size by natural interactions could be used for further personalization on mobile devices.
%The paper is organized as follows. We start with an overview of related work. We present our study design and procedure before showing our results. We conclude with a discussion and an outlook to future work.

\section{Related Work}
Parhi et al. performed a study about the error rates and preferred positions of different sized targets on a screen. They found out that as the target's size increases the error rate and time to reach decreases. Users also preferred the middle of the screen in contrast to the upper left and lower right corner \cite{parhi2006target}. Another study by Karlson et al. also demonstrates that the middle of the screen is the easiest to reach for a user while reaching the corners is more time consuming \cite{karlson2006studies}. Since the target's position seems to be relevant for a user's performance, we also tested different target positions in our study. Buschek and Alt also took the hand size into consideration and discovered that users with smaller hands show a larger y-offset in the upper left corner but are more accurate in the lower area of the screen when trying to hit a target \cite{buschek2015touchml}. In this study only one hand dimension was measured. We chose to measure four different dimension in order to find the highest correlations. A different study by Balakrishnan et al. investigated users' satisfaction for different keyboards. They tried to find correlations between the satisfaction and the user's thumb length or circumference. The results show that the circumreference correlates with the satisfaction of the key size and space between the keys \cite{balakrishnan2008study}. In contrast to Balakrishnan et al. we investigated the hand size in total but not the thumb size itself. Bergstrom et al. defined a functional area which is the reachable region of a user's thumb without repositioning his hand. Smaller hands also show a lesser functional area \cite{bergstrom2014modeling}. As observed by Boring et al. users reposition their hand in order to reach certain areas of the screen \cite{boring2012fat}. These findings are interesting factors which we also took into consideration for our study by measuring the radius of the participant's thumb and including tasks with predefined and undefined hand positions.

\begin{figure*}[!ht]
    \subfigure[Radius Task]{\includegraphics[width=0.16\textwidth]{figures/screenshot01.png}} 
    \subfigure[Tapping Task]{\includegraphics[width=0.16\textwidth]{figures/screenshot02.png}}
    \subfigure[Scrolling Task]{\includegraphics[width=0.16\textwidth]{figures/screenshot03.png}} 
    \subfigure[Swiping Task]{\includegraphics[width=0.16\textwidth]{figures/screenshot04.png}} 
    \subfigure[Maximum Zooming Task]{\includegraphics[width=0.16\textwidth]{figures/screenshot05.png}} 
    \subfigure[Frame Zooming Task]{\includegraphics[width=0.16\textwidth]{figures/screenshot06.png}} 
\caption{Interaction tasks} 
\end{figure*}

\section{Study}
In order to find correlations between a user's hand size and his mobile touch interactions we implemented an android application to measure data for different interactions. 

\subsection{Study Design}
%Since we wanted the measured data to be as natural as possible we chose to investigate four main gestures which users have to perform regularly when operating a smartphone: tapping, swiping, scrolling and zooming. The latter was tested in two tasks which concludes in five tasks in total. The user was allowed to adapt his hand position while performing these tasks. At the beginning of our exploratory study it was unclear whether this natural behaviour could deliver any useful results at all. This is why we chose to add a sixth unnatural task for which the hand position was predetermined. The radius of the user's thumb was meausured while he was holding the phone next to the heel of the hand. The specific tasks were designed as followed:
The study design was within subjects so all participants performed the following tasks (see figure 1):
\begin{itemize}
	\item{Radius Task:} The user was supposed to swipe a quarter circle from the right edge of the screen to the lower left
	\item{Tapping Task:} The user was instructed to hit small crosses on the screen as pecisely as possible. 144 crosses appeared in an randomized order
	\item{Scrolling Task:} The user had to scroll a list from top to bottom
	\item{Swiping Task:} The user was supposed to swipe a slider from left to right. Four slider positions were tested: top, middle, bottom and diagonal
	\item{Maximum Zooming Task:} The user had to zoom a blue rectangle as far as possible with one zoom gesture
	\item{Frame Zooming Task:} The user was instructed to zoom a blue rectangle to fit into a frame. Three frame sizes were tested: small, medium and large. The user was allowed to execute multiple zoom gestures
\end{itemize}
 

%The phone we used for our study was the HTC one max with a screen size of 5.9 inches. The study design was within subjects so all participants performed all of the tasks. The first one always was the unnatural interaction followed by the rest in an randomized order. It was also neccessary to measure the participants' hands in order to compare the hand sizes to the measured data delivered by our application. We chose to measure the participants' hand length, width, total span (from thumb to pinky finger) and zooming span (from thumb to index finger) in order to have more options for possible correlations. 
We measured the participants' hand length, width, total span (from thumb to pinky finger) and zooming span (from thumb to index finger). Figure 2 shows these measured dimensions.

\begin{figure*}[h!]
	\subfigure[Total Span]{\includegraphics[width=0.24\textwidth]{figures/hand01cut}} 
    \subfigure[Hand Length]{\includegraphics[width=0.24\textwidth]{figures/hand02cut}} 
    \subfigure[Hand Width]{\includegraphics[width=0.24\textwidth]{figures/hand03cut}} 
    \subfigure[Zomming Span]{\includegraphics[width=0.24\textwidth]{figures/hand04cut}} 
\caption{Hand Measurements}\label{fig:hand}
\end{figure*} 

\subsection{App}
The android application implemented for this study contained a screen for entering data about the user (hand measurements, age and gender). Before each task some instructions about the assignment were displayed. While performing the tasks, the application tracked different features which were stored into a database. These comprised of the x- and y- positions of the touch events and sensor data about acceleration, rotation and orientation. Task specific features were the number of scrolls, zooming span, timestamps etc. The phone we used for our study was the HTC one max with a screen size of 5.9 inches.

\subsection{Participants}
The 62 participants of this study were 36 males and 26 females between 18 and 36 years old. The average age was 24. Their hand lengths differed between 152 and 224 mm.

\subsection{Procedure}
% Studienaufbau, WIE
The participants have been invited to a 15 minute time slot to take part in our hand measurement study. They could receive credits or an amazon voucher for their participation. In order to allow the usage of their hand and touch data, they had to sign a letter of agreement.\\
At first, the participants hand dimensions were measured manually as described before (see figure \ref{fig:hand}). Their data was then entered directly into our app. The participants started with the artificial Radius Task in a predetermined hand position. After that, the natural tasks came up in a randomized order according to a latin square. The participants were free in solving the tasks, except they were only allowed to use one hand. For the zooming tasks, participants were instructed to use the other hand as well or to leave the device on the table.

\begin{table}[ht]
\centering
\caption{Hand size correlations}
\label{handSizeCorrelations}
\begin{tabular}{ll|llll}
 &users  &total span  &zoom span  &length  &width  \\ \hline
 &total span  &  &0.74  &0.705  &0.754 \\
 &zoom span  &0.764  &  &0.772  &0.669 \\
 &length  &0.702  &0.757  &  &0.818 \\
 &width  &0.776  &0.673  &0.841  &
\end{tabular}
\end{table}

\section{Results}
To understand the results of our study we first have to look at the measured hand data. The correlation of the different hand measurements is not as high as a naive assumption would suggest. The weaker correlations out of the four measured are the total-span with the zoom-span. The tighter measurements were therefore the width and the length of the hand (0.841, see Table \ref{handSizeCorrelations}). As the different measurements aren't linked tight enough we decided to compare the results with all of the four measurements separately. All results can be found in Table \ref{allCorrelations}. A calculation of significance could not be done because this study is only exploratory. The highest correlations happened in the Radius Task with the fixed hand position. We got the best result looking at the minimal X value reached there. Participants with a bigger hand, a bigger Zoom Span in particular, were able to reach a smaller X value. The origin of mobile phone coordinate systems is on the top left, so this is equivalent to how far the subjects got to the left edge of the screen with their thumb. Figure \ref{plotRadius} shows the according data plot with the correlation of -0.57. Measuring the distance between the start and endpoint of the touch interaction showed a correlation of 0.45 regarding the Zoom Span again. Therefore participants with a higher Zoom Span had a higher distance between these two points. The according plot in figure \ref{plotRadiusStartEnd} looks fairly consistent with no major outliers.\\
Another interesting result is the number of scrolls the participants needed to scroll to the end of the list in the Scrolling Task. Subjects with a wider hand needed less scrolls here. While observing the task operated by the participants we could notice that most subjects flicked the list and then waited a short moment until the list slowed down spinning. But some subjects just dragged the list and did not flick it at all, which is how the top outliers in the according figure \ref{plotScrolling} can be explained.\\
Lower in correlation but also an interesting plot is from the zooming task, where the subjects had to perform the widest zoom gesture they could. The highest correlation is with the hand width. It is important to know that most participants were able to reach to the edges of the devices screen which probably lowered the correlation. Looking at the plot (figure \ref{plotZooming}) we see that not all the recorded data is plausible because the span of the lowest point is around 5cm. We did remove some of these points to check their influence and got an even higher correlation.\\
Other interesting but not as high correlations were also found in the Tapping Task regarding the x-orientation of the touches on the top left quarter of the screen. Compared with the with of the hand we got a result of -0.32\\
At the Swiping Task we got the lowest results overall with the highest value at 0.3 regarding the maximal y-difference happening during the slide on the bottom slider.

\begin{figure}[ht]
	\centering
  \includegraphics[width=0.5\textwidth]{figures/plotRadius.png}
	\caption{Radius task: Minimal reached X value.}
	\label{plotRadius}
\end{figure}

\begin{figure}[ht]
	\centering
  \includegraphics[width=0.5\textwidth]{figures/plotScrolling.png}
	\caption{Scrolling task: Number of scrolls.}
	\label{plotScrolling}
\end{figure}

\begin{figure}[ht]
	\centering
  \includegraphics[width=0.5\textwidth]{figures/plotZooming02.png}
	\caption{Zooming task: Span of maximal zooming gesture.}
	\label{plotZooming}
\end{figure}

\begin{figure}[ht]
	\centering
  \includegraphics[width=0.5\textwidth]{figures/plotRadiusStartEnd.png}
	\caption{Radius task: Distance between starting and ending point of touch.}
	\label{plotRadiusStartEnd}
\end{figure}


\begin{table}[ht]
		\caption{Correlations by Task (T): 1: Tapping, 2: Scrolling, 3: Sliding, 4: Radius, 5: Zooming }
		\label{allCorrelations}
    \begin{tabular}{|lll|rrrr|}
    \hline
    T 	& Notes       & Value     & T.Sp  & Z.Sp   & L. & W. \\ \hline
    1    & 1/4 Scr.  	& Time diff & -0.1  & -0.02  & 0.15   & -0.08 \\
    ~    & Whole S.   & "         & -0.1  & 0.04   & 0.17   & -0.06 \\
    ~    & 1/4 Scr.  	& X-der     & -0.23 & -0.09  & -0.17  & -0.2  \\
    ~    & Whole S.   & "         & -0.21 & -0.1   & -0.2   & -0.17 \\
    ~    & 1/4 Scr.  	&  Y-der    & -0.16 & 0      & -0.11  & -0.16 \\
    ~    & Whole S.   & "         & -0.05 & 0.04   & -0.06  & 0.07  \\
    ~    & 1/4 S.  		& Pressure  & 0.21  & 0.14   & 0.07   & 0.19  \\
    ~    & Whole S.   & "         & 0.29  & 0.2    & 0.14   & 0.22  \\
    ~    & 1/4 Scr.  	& Orient.X  & -0.5  & -0.21  & -0.13  & -0.32 \\
    ~    & Whole S.   & "         & -0.2  & -0.2   & -0.11  & -0.27 \\
    ~    & 1/4 Scr.  	& Orient.Y  & -0.04 & -0.06  & 0.15   & -0.02 \\
    ~    & Whole S.   & "         & -0.06 & -0.11  & 0.09   & -0.05 \\
    ~    & 1/4 Scr.  	& Size      & 0.21  & 0.17   & 0.1    & 0.24  \\
    ~    & Whole S.   & "         & 0.24  & 0.14   & 0.13   & 0.26  \\ \hline
    2    & ~          & Number    & -0.31 & -0.29  & -0.35  & \cellcolor[gray]{0.9}-0.43 \\
    ~    & ~          & Avr. dist & 0.13  & 0.04   & 0.05   & 0.11  \\
    ~    & ~          & Start X   & 0.04  & 0.08   & 0.07   & 0.09  \\
    ~    & ~          & Start Y   & 0     & -0.05  & -0.04  & 0.01  \\ \hline
    3    & top  			& End X     & -0.13 & -0.17  & -0.1   & -0.02 \\
    ~    & middle  		& "         & 0.13  & 0.09   & 0.06   & 0.06  \\
    ~    & bottom 		& "         & 0.26  & 0.16   & 0.17   & 0.22  \\
    ~    & diagonal 	& "         & 0.09  & -0.04  & 0.04   & 0.1   \\
    ~    & top  			& Y Diff.   & 0.08  & -0.06		& 0.18   & 0.25  \\
    ~    & middle  		& "         & 0.03  & 0.07   & 0.08   & 0.01  \\
    ~    & bottom 		& "         & 0.3   & 0.25   & 0.23   & 0.19  \\
    ~    & diagonal 	& "         & 0.11  & 0.17   & 0.08   & 0.16  \\ \hline
    4    & st.-end   	& dist      & 0.26  & \cellcolor[gray]{0.9}0.45   & 0.35   & 0.25  \\
    ~    & ~          & Min X     & -0.5  & \cellcolor[gray]{0.9}-0.57  & -0.52  & -0.48 \\
    ~    & ~          & Min Y     & -0.1  & -0.22  & -0.16  & -0.08 \\ \hline
    5    & span       & var.      & 0.21  & 0.12   & 0.28   & 0.2   \\
    ~    & span       & max       & 0.29  & 0.16   & 0.18   & \cellcolor[gray]{0.9}0.36  \\ \hline
    \end{tabular}
\end{table}

\subsection{Machine Learning}
After identifying the values with high correlation to any of the user's hand dimensions, we applied machine learning techniques in order to predict small and big hands from the measured data.\\
At first, the user data was sorted by hand length and filtered in such a way that around 20\% of the biggest and the smallest hand sizes are in our data for machine learning, which makes 12 users of each class and a total of 24 data sets.\\
The feature set was built upon the values with the highest correlation, which are the following as shown before:\\ Scrolling Number, Radius Minimum X, Taps 1/4 Screen Derivation, Taps 1/4 Screen Orientation, Zooming Maximum Span (see table \ref{allCorrelations}). These features where labeled with 'B' and 'S' for the 12 biggest and the 12 smallest user hands.\\
With this data set, a 10-fold cross validation was performed. As classifier, a Random Forest with default parameters was applied.\footnote{\url{http://scikit-learn.org/stable/modules/ensemble.html\#random-forests}} We could achieve a score of 0.75 with this data set.\\
Using single features of this data set for a cross validation, the features Taps 1/4 Screen Orientation (0.675) and Radius Minimum X (0.65) reach the highest score and seem to influence the overall result most. This shows a high potential for predicting the user's hand size from touch interaction. Once again, there is a big advantage in the result with predetermined hand position. To predict the users hand size properly, tasks with fixed hand position may be used for measurement.

\section{Discussion}
Investigating hand size in conjunction with mobile touch interactions is difficult. People have their own habits interacting with their devices, regardless of their hand size. Especially with a foreign, oversized device like the one we used in our study, people behave in different ways. People with big hands had in fact no troubles to reach every task stimuli on the screen. In contrast, people with smaller hands tended to compensate their deficits in moving their hand across the device to reach far corners. Observing this difference in reality seems quite easy, while distinguishing it in our data was hardly possible.\\
% von jonas: hier gut eingefügt oder unterbricht der teil zu stark?
Looking at the results some correlations are quite unexpected. The maximum zooming span (Zooming Task) p.e. had a much higher correlation with the width of the participants than the zooming span. We would have suggested that the correlation is highest with a measurement involving the thumb. This is probably because of the measurement technique we used. For the Zoom Span the Subjects had to spread and press their thumb and index finger on a table. Some participants pressed harder than others and some were physically able to spread their fingers wider. This explains the comparable low correlations between these hand measurements and also some unexpectedly low correlations at the tasks. Because of this we have to take the correlations with the Total Span and the Zooming Span with care.\\
% einfügen ende
Finding out the user's hand size automatically might only be possible by an unnatural, artificial task. As the most promising result we got was from the Radius Task, this might be the right way to investigate hand size. Our correlations as well as the classification via random forest showed that a predetermined hand position delivers a consistently higher score.\\
For the user, it might not be that usable to perform some artificial task for predicting his hand size. It would have been even more comfortable to obtain the hand size inconspicuously from natural, frequent interactions like we tried to evaluate with all our other tasks. It turned out that a fixed hand position seems to be necessary for a clear result. After all, it could still be beneficial for the user to let his device know about his hand size, by any kind of interaction task or measurement. It could then be an important personalization factor and enhance mobile touch interaction.


\section{Conclusion and Future Work}
We came up with different challenges when investigating hand size and mobile touch interactions. In our study, it was hardly possible to determine the user's hand size from his mobile touch interactions during our tasks.\\
We could show a higher correlation between touch interaction and hand size when we determined the hand position. As this might be uncomfortable for users, the other tasks in our study were designed in such a way that hand position is free and left over to the user in order to keep the data as natural as possible. It turned out that machine learning techniques can help to predict the user's hand size but once more a fixed hand position delivers a higher score.\\
Further investigations should eventually determine hand positions in order to evaluate the user's hand size. As the most promising result we got was from our radius task, this interaction could eventually be used to predict the user's hand size.\\ 
%hier nochmal results zusammen fassen?!
Knowing about the user's hand size could then enhance mobile interaction. Another personalization aspect could be established through hand dimensions. If our mobile devices know about our hand sizes, they could adapt the user interfaces in order to facilitate the interaction. Navigation bars and other important UI Elements could be moved to make them more reachable. Thereby, our device would be even smarter in adapting to our personal needs.

%\begin{itemize}
%\item Conclusion: am vielversprechendsten ist vermutlich ... Wie würde man speziell dieses noch in neuer Studie untersuchen
%\item weitere Daten angucken und evaluieren
%\item make mobile interaction smarter (Bezug zur Introduction nehmen)
%\end{itemize}



% Balancing columns in a ref list is a bit of a pain because you
% either use a hack like flushend or balance, or manually insert
% a column break.  http://www.tex.ac.uk/cgi-bin/texfaq2html?label=balance
% multicols doesn't work because we're already in two-column mode,
% and flushend isn't awesome, so I choose balance.  See this
% for more info: http://cs.brown.edu/system/software/latex/doc/balance.pdf
%
% Note that in a perfect world balance wants to be in the first
% column of the last page.
%
% If balance doesn't work for you, you can remove that and
% hard-code a column break into the bbl file right before you
% submit:
%
% http://stackoverflow.com/questions/2149854/how-to-manually-equalize-columns-
% in-an-ieee-paper-if-using-bibtex
%
% Or, just remove \balance and give up on balancing the last page.
%
%\balance{}


% BALANCE COLUMNS
\balance{}

% REFERENCES FORMAT
% References must be the same font size as other body text.
\bibliographystyle{SIGCHI-Reference-Format}

%\bibliographystyle{unsrtnat}
\bibliography{literature}
%\bibliographystyle{SIGCHI-Reference-Format}

\end{document}

%%% Local Variables:
%%% mode: latex
%%% TeX-master: t
%%% End:
